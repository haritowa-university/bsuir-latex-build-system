\sectioncentered*{Введение}
\addcontentsline{toc}{section}{Введение}
\label{sec:introduction}

Люди давно поняли, что управление временем, задачами и контактами -- одни из самых важных условий повышения личной эффективности. Наука управления временем объединяет всевозможные техники и приемы, позволяющие беречь драгоценное время, распределять его более рационально и благодаря этому быстрее достигать своих целей \cite{time_management}. Управление задачами представляет собой их систематическое фиксирование, ранжирование по приоритету, выполнение, контроль статуса и истории их выполнения \cite{task_management}. Построение сети персональных контактов способствует поискам работы и продвижению по карьерной лестнице. 

Перечисленные задачи особенно актуальны для людей, участвующих в образовательном процессе в ВУЗах: для студентов, преподавателей, а также для работников деканатов и других структурных подразделений университетов. У студентов и преподавателей практически на каждый день в расписании запланировано несколько занятий, в любое время у них есть задачи, которые они обязаны выполнить, студенты постоянно взаимодействуют с большим числом преподавателей, у которых в свою очередь есть еще большее число студентов: даже просто запомнить такое количество имен и лиц практически невозможно. Но на данный момент не существует решения, которое бы объединяло всю вышеперечисленную функциональность. И людям для выполнения своих задач приходится комбинировать существующие и приспосабливать под свои нужды. Каждый человек вынужден тратить своё время на поиск средств, тратить время на их конфигурирование. Возникает разрозненность способов, которая приводит к тому, что другим людям также приходится тратить своё время на адаптацию своего набора средств к другим людям. 

Для этих задач реализовано огромное число приложений для мобильных, настольных операционных систем, большое число  веб-приложений; кроме того, люди по-прежнему продолжают использовать ежедневники, записные книги и бумажные календари.

Целью настоящего дипломного проекта является разработка программного средства, которое в рамках ВУЗа смогло бы предложить участникам учебного процесса унифицированные способы управления календарём, задачами, а также предоставляет канал обмена информацией между студентами и преподавателями. 

В пояснительной записке к дипломному проекту излагаются детали поэтапной разработки приложения повышения профессиональной эффективности в процессе обучения. В первом разделе приведены результаты анализа литературных источников по теме дипломного проекта, рассмотрены особенности существующих систем-аналогов, выдвинуты требования к проектируемому ПС. Во втором разделе приведено описание функциональности проектируемого ПС, представлена спецификация функциональных требований. В третьем разделе приведены детали проектирования и конструирования ПС. Результатом этапа конструирования является функционирующее программное средство. В четвертом разделе представлены доказательства того, что спроектированное ПС работает в соответствии с выдвинутыми требованиями спецификации. В пятом разделе приведены сведения по развертыванию и запуску ПС, указаны требуемые аппаратные и программные средства. Обоснование целесообразность создания программного средства с технико-экономической точки зрения приведено в шестом разделе. Итоги проектирования, конструирования программного средства, а также соответствующие выводы приведены в заключении.
